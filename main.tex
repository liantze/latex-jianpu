\documentclass[11pt,fontset=ubuntu]{ctexart}
\usepackage{geometry}
\geometry{hmargin=3cm,vmargin=2.5cm}
\usepackage{jianpu}
\usepackage[many]{tcolorbox}
\tcbuselibrary{skins}
\tcbuselibrary{minted}
\usepackage{FiraMono}
\usepackage[lf]{FiraSans}

\title{一个简陋的简谱宏包}
\author{}
\date{}
\begin{document}
\maketitle

很笨,处理方法很笨很笨。每个token都是在一个盒子里(默认宽度1em)。这样就不用头痛下划线要完美衔接的问题。够笨了。但算了,我就是一条得过且过的咸鱼。还在练习\LaTeX3,不是很熟练。

\begin{tcblisting}{breakable}
\begin{jianpu}
\sffamily 5 - {6_} {6_}  | 5 - 3 | \slurB 5 {3_} {2_} \slurE 1 |
{5,} - - | {6,} - 1 | 2 - 5 | 3 -- | \\
\rmfamily 蓝~蓝 的~天~空~银~~ 河~里 ~ ~ ~  有~只~小~白~船

\sffamily 5 - 6  | 5 - 3 | 5 {3_} {2_} 1 | {5,} - - |
{6,} - 1 | {5,} - 2 | 1 -- | \\
\rmfamily 船~上~有~颗~桂~~花~ 树 ~ ~ ~ 白~兔~在~游~玩

\sffamily 3- 3 | 3- 2 | 3- 6 | 5 - - | 3 - 2 | 3 - 6 | 5 -- |\\
\rmfamily 桨~儿~桨~儿~看~不~见~~~船~上~也~没~帆

\sffamily {1'} - - | 5 - - | \slurB {3} - {5} \slurE |
6 - - | \slurB 5 3 1 \slurE | \slurB {5,} - 2 \slurE | 1 - - {||}\\
\rmfamily 飘 ~~~ 呀 ~~~ 飘~~~呀~~~ 飘~向~西~~~天
\end{jianpu}
\end{tcblisting}


想要加歌词需要自己加\verb|~~|强行插入空格。没有自动对齐,手动\&对齐也不会有。

因为使用统一宽度的盒子所以目前只适合中文歌词(至少是CJK)。

\begin{tcblisting}{}
\begin{jianpu}
{0_} {5,_} \; {1_} {2_} | 3 {3_} {3_} 4 4 | 5
\slurB {2_} {3_} \slurE 3. {2=} {2=} |
2 {1_} \slurB {7,_} 1 \slurE .  {3=}{3=} | 3 {2_} 2. \\
~い \; とお~しいひとの~ため~に~ いま~なにがー~でき~るかな
\end{jianpu}
\end{tcblisting}

可以改变盒子宽度\verb|\setjpnotewidth{1.2\ccwd}|

\begin{tcblisting}{}
\setjpnotewidth{1.2\ccwd}
\begin{jianpu}
3 {6,} \; {#5,_} {6,_} {7,_} {2_} | 1 {6,} \; {7,=} {1_} {2_} {4_} | 3 . {3_} \; {2_} {3_} {4_} {6_} | {#5} 4 3 2 | \\
叮咚\;我有一个~秘密\;悄悄告诉~你~欢\;迎你来到~天堂入口
\end{jianpu}
\end{tcblisting}

但是不便在简谱当中随机改变!所以才说不适合CJK以外语言的歌词。如下,太难看了、太麻烦了。

\begin{tcblisting}{}
\setjpnotewidth{1.5em}
\begin{jianpu}
1. {2_} 3. {1_} | 3 {[2.75em]1} 3 - | \\
\small {Doe} , a {deer} , a ~ {fe-} {[2.75em]male} {[2.5em]deer}
\end{jianpu}
\end{tcblisting}

“行间”简谱不会有歌词的吧,就不需要那么宽的盒子,默认宽度0.75em。有升降号的符则至少1em。

\begin{tcblisting}{}
行间简谱,有时只是要拿一段旋律\jianpuInline{{5_} {6_} {1'} {3'} {#2'} {3'} {b2'} {n2'} }出来说说
\end{tcblisting}

装饰音,我能想到的处理方式暂时只有这样了……难看,难用!

\begin{tcblisting}{}
\begin{jianpu}
1 2 {\slurB[0.2ex][0.25em]} {^b3} 2 \slurE 1 {\slurB[2pt]} {1'} {1''} {\slurE[5pt]} {6,_} {6,,=}
\end{jianpu}
\end{tcblisting}

\end{document}
